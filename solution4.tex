\documentclass[a4papper]{article}

\usepackage[final]{pdfpages}
\usepackage{graphicx}
\usepackage{amssymb, amsmath, amsthm, mathrsfs}
\usepackage{indentfirst}
\usepackage{amsmath}
\usepackage{listings}
\usepackage{amsfonts}
\newtheoremstyle{neosn}{0.5\topsep}{0.5\topsep}{\rm}{}{\sc}{.}{ }{\thmname{#1}\thmnumber{ #2}\thmnote{ {\mdseries#3}}}
\theoremstyle{neosn}
\newtheorem{problem}{Problem}

\begin{document}
    \begin{center}
    {\bf Homework 3} \\
        \today \\
        TangLin
    \end{center}

    \problem{Construct formula of language $(+,0,1,=)$ whose length is $O(\log n)$ with free variable $x$ for any
        natural number $n$, and it is true in Module $(\mathbb{N}, +, 0, 1)$ if and only if $x = n$.}
    \\

    Solution: \\

    \\
    Let the formula which we want to build be $R_n(x)$.
    We can build it inductively.
    First, let
    \[
        R_0(x) = (x=0)
    \]
    then, for any natural number $n \leqslant 1$,
    \[
        R_n(y) = \begin{cases}
                     \exists x \left(R_{\frac{n}{2}}(x) \land y=x+x\right), & \text{if } n \text{ is even number.} \\
                     \exists x \left(R_{n-1}(x) \land y=x+1\right), & \text{if } n \text{ is odd number.}
                 \end{cases}
    \]

    Obviously, it is true for all natural number $n$
    \[
        R_n(x) = 1 \Leftrightarrow x=n.
    \]
    Now we need to analyse its number of symbols.
    We use function $l(R)$ to indicate the number of symbols of formula $R$.
    $l(R_1(x)) = 5$, when $n=0$; $l(R_n(x)) = l(R_{\text{previous}}(x)) + 10$ when $n$ is odd or even.
    So, we can get the relation
    \[
        \begin{array}{ll}
            l(R_n(x)) &= l(R_{\lfloor \frac{n}{2} \rfloor}(x)) + 10 \\
                        &= l(R_{\lfloor \frac{n}{2^2} \rfloor}(x)) + 2 \cdot 10 \\
                        & = l(R_{\lfloor \frac{n}{2^3} \rfloor}(x)) + 3 \cdot 10 \\
                        & = \cdots \\
                        & = l(R_0(x)) + (\log n + 1)\cdot 10 \\
                        & = 10 \log n + 15 \in O(\log n).
        \end{array}
    \]
    So, we design a formula $R_n(x)$ with number of symbols $O(\log n)$ which can decide $x = n$ and
    the number of variables are $O(1)$.
    \\

    \problem{Construct formula of language $(+,0,1,=)$ whose length is $O(n)$ with three free variables
    $x,y,z$, and it is true in Module $(\mathbb{N}, +, 0, 1, =)$ if and only if $x = y \cdot z$ and $y < 2^n$.}

    \\

    Solution: \\

    Let the formula which we want to find be $P_n(x,y,z)$.
    We can also build it inductively.
    First, let
    \[
        P_0(x,y,z) = (y=0 \land z = 0) \lor (y=1 \land z=x).
    \]
    Then we are building the inductive part.\\
    Because we have the limiting condition $y < 2^n$, so we have to design the recursive
    process from $n$ to $n+1$ such that $y$ satisfies corresponding condition.
    We can do it, let
    \[
        y = y_1 + y_2 + 1
    \]
    where $y<2^{n+1})$ and $y_1, y_2 < 2^n$.
    So the maximum of $y+1$ is
    \[
        1+y = y_1 + y_2 + 1+1 = 2^n-1+2^n-1+2 = 2^{n+1}
    \]
    i.e. $y$ has maximum $2^{n+1} -1$, it satisfies our requirement.
    For any $z$, we have $zy = z(y_1+y_2+1) = zy_1+zy_2+z$.
    Then we build the inductive formula,
    \[
        P_{n+1}(x,y,z) = \exists y_1, y_2, r, s \left(y = y_1+y_2+1 \land x = r+s+z
        \land P_n(r,y_1,z) \land P_n(s,y_2,z)\right)
    \]
    But there are two $P_n$ occurrences, and the total length will be the exponential of $n$.
    So we have to do some transformation.
    \[
        \begin{array}{ll}

            P_{n+1}(x,y,z) = & \exists y_1, y_2, r, s \Big{(} y = y_1+y_2+1 \land x = r+s+z \land \\
            & \land \forall t \exists u (t=y_1 \land u = r) \lor (t=y_2 \land u=s) \to P_n(u,t,z) \Big{)}
        \end{array}
    \]
    Now we analyse the number of symbols of formula $P_n(x,y,z)$.
    We use notation $l(P_n)$ to indicate the number of symbols of $P_n$.
    \[
        l(P_0(x,y,z)) = 19
    \]
    then for any $n>=1$,
    \[
        l(P_n(x,y,z)) = l(P_{n-1}(x,y,z)) + 50
    \]
    So
    \[
        l(P_n(x,y,z)) = 19+50n \in O(n).
    \]

    We have built the formula which satisfies the condition with 9 variables.
    \\





\end{document}