\documentclass[a4papper]{article}

\usepackage[final]{pdfpages}
\usepackage{graphicx}
\usepackage{amssymb, amsmath, amsthm, mathrsfs}
\usepackage{indentfirst}
\usepackage{amsmath}
\usepackage{listings}
\usepackage{amsfonts}
\usepackage{wasysym}
\usepackage{stmaryrd}
\newtheoremstyle{neosn}{0.5\topsep}{0.5\topsep}{\rm}{}{\sc}{.}{ }{\thmname{#1}\thmnumber{ #2}\thmnote{ {\mdseries#3}}}
\theoremstyle{neosn}
\newtheorem{problem}{Problem}

\begin{document}
    \begin{center}
    {\bf Homework 3} \\
        \today \\
        TangLin
    \end{center}

    \problem{Prove there is \textbf{PSPACE}-complete problem.}\\

    Solution: \\

    \\

    We will consider the language \textbf{TQBF} which is true quantified boolean formulae with form
    \[
        Q_1x_1Q_2x_2\cdots Q_n x_n \varphi(x_1, x_2, \cdots, x_n)
    \]
    where $Q_i$ is $\forall$ or $\exists$ and $\varphi$ is a boolean formulae with $n$ variables.
    So,
    \[
        \textbf{TQBF} = \{\psi = 1| \psi = Q_1x_1Q_2x_2\cdots Q_n x_n \varphi(x_1, x_2, \cdots, x_n)\}
    \]

    First, we show that \textbf{TQBF} $\in$ \textbf{PSPACE}.
    Let
    \[
        \psi = Q_1x_1Q_2x_2\cdots Q_n x_n \varphi(x_1, x_2, \cdots, x_n)
    \]
    be a quantified boolean formulae with $n$ variables and the size of $\varphi$ is $m$.
    Obviously, if all the variables are assigned then we need only $O(m)$ space to decide it.
    Now we will construct a recursive algorithm to decide $\psi$.
    We will use notation $\psi|_{x_i = b}$ to denote that replacing all the occurrences of variable
    $x_i$ with $b \in \{0,1\}$ with dropping quantifier $Q_i$. \\
    Algorithm will work as follows:\\
    \begin{enumerate}
        \item if $Q_i = \exists$ then output 1 if and only if at least one of $\psi|_{x_i=0}$ and
        $\psi|{x_i = 1}$ returns 1.
        \item if $Q_i = \forall$ then output 1 if and only if $\psi|_{x_i=0}$ and
        $\psi|{x_i = 1}$ both return 1.
    \end{enumerate}
    Obviously, the recursive algorithm conform to the definition of $\exists$ and $\forall$, so
    it does indeed return the correct answer on any formula $\psi$. \\
    Whenever we drop a quantifier, we need copy the formula $\varphi$ once (we can reuse the space
    to decide $\psi|_{x_i=1}$ and $\psi | _{x_i = 0}$). \\
    Let $s_{n,m}$ denote the space of the recursive algorithm uses on $\psi$ with $n$ variables.
    We can reduce one variable when we drop a quantifier, so
    \[
        s_{n,m} = s_{n-1,m} + O(m)
    \]
    and we can get $s_{n,m} = O(n\cdot m)$, obviously, $n < m$ and we can say that
    the QBF $\psi$ can be decided in $O(m^2)$ space. \\
    So \textbf{TQBF} \in \textbf{PSPACE}. \\

    \\

    Now we will show that for every $L \in$ \textbf{PSPACE}, $L \leqslant_p$ \textbf{TQBF}.
    Let $M$ be a TM that can decides $L$ in $S(n)$ space and let $x \in \{0,1\}^n$.
    First, we show that there is a Boolean formula $\varphi_{M,x}$ such that for every two
    strings $C, C'$ which encode the configuration of $M$, and $\varphi_{M,x}(C,C') = 1$ if and
    only if $C$ and $C'$ are valid adjacent configurations in the configuration graph of $G_{M,x}$. \\
    We can consider all the possibilities of configurations, the number if $2^{O(S(n))}$, it is too
    large to be express in CNF or DNF\@.
    We can group them by the head position, i.e. we can group them to $O(S(n))$ groups like
    \begin{figure}[h]
        \centering
        \begin{tabular}{|c|c|c|}
            \hline
            $C$ & $C'$ & \varphi_{M,x}(C,C') \\
            \hline
            \cdots - - - * - - - \cdots & \cdots - * - - - - - - \cdots & 0 \\
            \cdots - - - * - - - \cdots & \cdots - - * - - - - - \cdots & 1 \\
            \vdots & \vdots & \vdots \\
            \hline
            \cdots - - - - * - - \cdots & \cdots - - - - - - * \cdots & 0 \\
            \cdots - - - - * - - \cdots & \cdots - - - - * - - \cdots & 1 \\
            \vdots & \vdots & \vdots \\
            \hline
            \vdots & \vdots & \vdots \\
        \end{tabular}\caption{Table of $\varphi$,``-'' represents a work tape cell and ``*''
        represents the head position.}
        \label{fig:figure}
    \end{figure}

    The first two columns are all possible configurations of $M$ and the third column is the result of
    $\varphi$.
    For every group we have to decide the all the possible cases, so the Table has $2^{2 O(S(n))}$ rows.
    In every group there will be few rows such that $\varphi = 1$, because $M$ can only move at most 1 step
    and modify at most 1 cell.
    In fact, there are at most 6 rows such that $\varphi = 1$, because the head will move to left or right
    or stay, and the cell can be modified or not, they are at most 6 different cases. \\
    The 6 different cases involve only 1 variable for the work cell, constant number of variables for
    states which decided by $M$.
    So for every group we can build a CNF with constant size, and the for the whole configurations we can
    get a CNF $\varphi$ with $O(S(n))$ size. \\
    Assume that $C$ need $m$ space.
    Now we will use $\varphi_{M,x}$ to come up with a polynomial space quantified boolean formula $\psi'$
    such that for every $C,C' \in \{0,1\}^m$, $\psi'(C,C') = 1 \Leftrightarrow$ there is a
    directed path from $C$ to $C'$ in $G_{M,x}$.
    Let $C=C_{start}$ and $C'=C_{accept}$, will get a quantified boolean formula $\psi = 1 \Leftrightarrow M$
    accepts $x$.
    We will define $\psi$ inductively.\\
    Let $\psi_i(C,C')$ be ture iff there is a path of length which does not exceed $2^i$ from
    $C$ to $C'$ in $G_{M,x}$.
    We have $\psi' = \psi_m$ and $\psi_0 = \varphi_{M,x}$.
    We can know that if $\psi_i = 1$ if and only if there is a configuration $C''$ such that $\psi_{i-1} = 1$,
    so we get
    \[
        \psi_i(C,C') = \exists C'' \psi_{i-1}(C, C') \land \psi_{i-1}(C'', C).
    \]
    But there occur two $\psi_{i-1}$, and the size of $\psi_m$ will be $2^m$, it is not polynomial, so
    we have to transform it to an equivalent form.
    Consider
    \[
        \psi_i = \exists C'' \forall D_1 \forall D_2 \big{(}(D_1 = C \land D_2 = C') \lor (D_1 = C'
        \land D_2 = C'') \big{)} \rightarrow \psi_{i-1}(D_1,D_2).
    \]
    We can easily transform $\rightarrow$ to $\land, \lor, \neg$, so this form has only one $\psi_{i-1}$.
    \[
        size(\psi_i) = size(\psi_{i-1}) + O(m)
    \]
    and then
    \[
        size(\psi) = size(\psi_m) = O(m \cdot m) = O(m^2).
    \]

    So we get a QBF with size $O(m^2)$, and the original problem has been reduced to \textbf{TQBF}.\\
    So \textbf{TQBF} is \textbf{PSPACE}-complete.


    \\

    \\

    \\
    \problem{Prove that the problem satisfiable formula in CNF can reduce by Karp to problem
    Integer Linear Programming (under a given system of linear inequalities with integer coefficients,
        find out whether it has a solution in integers).}

    \\

    Solution: \\

    \\
    Assume that the CNF has form:
    \[
        \varphi(x_1, x_2, \cdots , x_n) = \bigwedge_{i} \left( \bigvee_j v_{i_j} \right),
    \]
    where $v_{i_j}$ is some variable $x_k$ or its negation $\neg x_k$.
    It is easily to express a CNF formula to an integer program:
    \begin{enumerate}
        \item First we add the constraints $0 \leqslant x_i \leqslant 1$ for every $i$ to
        ensure that the variables can only be assigned with $0$ or $1$.
        \item Then we will express every clause of CNF to an integer inequality.
        for any variables $x_i$ and $x_j$
        \[
            \begin{array}{c}

            x_i \lor x_j \Leftrightarrow x_i + x_j \geqslant 1. \\
            \neg x_i \lor x_j \Leftrightarrow (1-x_i) + x_j \geqslant 1. \\
            x_i \lor \neg x_j \Leftrightarrow x_i+(1-x_j) \geqslant 1.\\
            \neg x_i \lor \neg x_j \Leftrightarrow (1-x_i) + (1 - x_j) \geqslant 1.
            \end{array}
        \]
        So for any number of variables connected by $\lor$ we can get the inequality inductively, because
        for any formula which are connected by $\lor$ obey the same laws.
        \item Then we collect all the clauses and get a system of inequalities, because all the clauses
        are connected by $\land$, and it means that the system has to satisfy all the clause.
    \end{enumerate}

    We have reduced the SAT problem to Integer Programming. \\

    Obviously, we can valid whether a given vector is a solution to the system in polynomial time,
    so Integer Programming is in \textbf{NP}.
    And the reduction indicates that SAT $\leqslant_p$ IP, so IP is NP-complete.

\end{document}