\documentclass[a4papper]{article}

\usepackage[final]{pdfpages}
\usepackage{graphicx}
\usepackage{amssymb, amsmath, amsthm, mathrsfs}
\usepackage{indentfirst}
\usepackage{amsmath}
\usepackage{listings}
\usepackage{amsfonts}
\usepackage{wasysym}
\usepackage{stmaryrd}
\usepackage{xcolor}
\newtheoremstyle{neosn}{0.5\topsep}{0.5\topsep}{\rm}{}{\sc}{.}{ }{\thmname{#1}\thmnumber{ #2}\thmnote{ {\mdseries#3}}}
\theoremstyle{neosn}
\newtheorem{problem}{Problem}
\newtheorem{example}{Example}
\newtheorem{correction}{Correction}

\begin{document}
    \begin{center}
    {\bf Correction of Homework 6} \\
        \today \\
        TangLin
    \end{center}

    \correction{}
    \\
    Solution:\\
    \\
    We need also an additional proof about size of certificate.
    Assume that length of binary representation of algebraic expression $exp$ is $len(exp) = n$,
    we need to prove that there will be certificated whose length is $poly(n)$. \\
    I'm going to use the fundamental theorem of algebra to prove it. \\
    First, we observe the algebraic function with one variable.
    If it is an $n$-th order expression, we need at least $\log_2 n$ bits to describe it.
    Because when $f(x) = x^n$ or $f(x) = \underbrace{x\cdot x \cdot x \cdots x}_{n \text{ times}}$,
    it requires the least bits.
    That is, if $len(exp) = n$, the expression is at most $log_2 n$ degree.
    According to the fundamental theorem of algebra, we know that there will be at most n different
    values $\alpha$ such that $f(\alpha) = 0$.
    So we can test number $\alpha$ from $1$ to $2n$, if $\forall \alpha \in \{1,2,\cdots, 2n\} f(\alpha) = 0$,
    then we can assert that $f(x) = 0$. \\
    For multi-variables function, we can get the similar results. \\
    If $len(f(x_1, x_2, \cdots , x_m)) = n$, then assume the $degree(x_i) = p_i$, we have
    \[
        \sum_{i=1}^m len(p_i) = \sum_{i=1}^m \log_2 p_i \leqslant n
    \]
    If we concentrate on only variable $x_k$ and fix all the others, we will get a single variable
    function, and we can test number $\alpha_k$ from $1$ to $2p_k$.
    It needs at most $\log_2 p_k$ bits.
    So for all the $m$ variables, we can use vector$(\alpha_1, \alpha_2, \cdots, \alpha_m)$,
    where$\alpha_i \in \{1, 2, \cdots, p_i\}$, to build the certificates.
    Every that kind of certificate need at most $\sum_{i=1}^{m} \log_2 p_i <= n$ bits.\\
    So, we have proved that
    \[
        M(exp\#\vec{x}) = 1 \Leftrightarrow \exists \vec{x} exp(\vec{x}) \neq 0.
    \]
    Where $\len(\vec{x}) \in O(len(exp)) = O(n)$.

    \correction{}
    \\
    Solution: \\
    \\
    This bug can be repaired, but the process of construction of set family need to be changed.
    First, we need find out degree of every vertex.
    We can easily prove that it can be achieved in polynomial time.
    We need only traverse $E$ and record every $d(i)$, then select the min one. \\
    For example,$G=(V,E)$, where $V=\{v_1, v_2, v_3, v_4, v_5, v_6, v_7\}$ and
    $E=\{(v_1, v_2), (v_1, v_3), (v_1, v_4), (v_3, v_4), (v_3, v_5), (v_4, v_6), (v_4, v_7)\}$
    \[
        d(v_1) = 3, d(v_2) = 1, d(v_3) = 3, d(v_4) = 4, d(v_5) = d(v_6) = d(v_7) = 1
    \]
    And then sort them from smallest to largest, it can be completed in polynomial time.
    We got
    \[
        v_2, v_5, v_6, v_7, v_1, v_3, v_4.
    \]
    Then we build set for every vertex by following rule according to the sorted list of vertices:
    \begin{enumerate}
        \item For $v_i$, we build an empty $S_i$, then append $i$ to it.
        \item If vertex $v_j$ which connects to $v_i$ has been translated to set, then append $j$ to $S_i$.
        \item When traverse all the neighborhoods have been check, then return to the first step for the next vertex.
        \item When all the vertices have been translated, then we get the set family.
    \end{enumerate}

    For our example,
    \begin{enumerate}
        \item $S_2$ = \{2\}
        \item $S_5$ = \{5\}
        \item $S_6$ = \{6\}
        \item $S_7$ = \{7\}
        \item $S_1$ = \{1,2\}
        \item $S_3$ = \{3,1,5\}
        \item $S_4$ = \{4,3,6,7\}
    \end{enumerate}

    What needs illustration is that, if the graph $G$ has some disjoint subgraph or some isolated points
    (they can be treated as subgraphs), we can execute the translation algorithm for every subgraph.\\


\end{document}