\documentclass[a4papper]{article}

\usepackage[final]{pdfpages}
\usepackage{graphicx}
\usepackage{amssymb, amsmath, amsthm, mathrsfs}
\usepackage{indentfirst}
\usepackage{amsmath}
\usepackage{listings}
\newtheoremstyle{neosn}{0.5\topsep}{0.5\topsep}{\rm}{}{\sc}{.}{ }{\thmname{#1}\thmnumber{ #2}\thmnote{ {\mdseries#3}}}
\theoremstyle{neosn}
\newtheorem{problem}{Problem}

\begin{document}
    \begin{center}
    {\bf Homework 3} \\
        \today \\
        TangLin
    \end{center}

    \problem{Prove that there is a universal single-tape Turing Machine whose time complexity
    increases linear time for all the single-tape Turing Machine. More specifically, there is
    a single-tape Turing Machine $U$, for any other single-tape TM $M$ there is a number $c$
    and string $p$: $U(p\#x) = M(x)$ and $t_U(p\#x) \leqslant c(t_M(x) + |x|)$} for any $x$.\\

    Solution: \\

    This solution is similar to the theorem which you told us about simulating single-tape TM by
    double-tape TM\@.
    I make a slight modification and get a single-tape universal TM\@.
    I separate the unique tape to 4 areas including Program Area, Status Area, Input Area and Output
    Area\@.
    \begin{figure}[h]
        \centering
        \begin{tabular}{c|c|c|c|c|c|c|c|c}
            \hline
            \Lambda & Program & \# & Status & \# & Input & \# & Output & \Lambda \\
            \hline
        \end{tabular}\caption{Tape Construction of S-UTM.}
        \label{fig:figure}
    \end{figure}

    Now I explain how are these areas constituted. \\

    \begin{enumerate}
        \item Program Area.\\
            This Area is used to store the TM $M$ which we want to simulate, it means that
            there will store the Transition Table.
            We have to encode the Machine $M$ before storing.
            The most convenient way is encoding it by binary codes.
            Assume that
            \[
                |Q_M| = l, |\Sigma_M| = m
            \]
            Every transition has form
            \[
                pa \mapsto p'a'\Delta
            \]
            Where $p,p' \in Q_M, a,a' \in \Sigma_M, \Delta \in \{-1,0,+1\}$.
            So every transition can be encoded to binary codes with $2(\log_2 l + 1 + \log_2 m + 1)+1$ bits.
            And there are at most $2^{l+m}$ different transitions.
            So we will cost at most
            \[
                |P| = 2^{2(\log_2 l + \log_2 m +2) + 1} \cdot 2^{l+m} = 32l^2 m^2 2^{l+m}
            \]
            bits to encode $M$. \\

        \item Status Area.

            This area is used to store all the statuses of $M$ and indicates which is the current status.
            Every status need at most $\log_2 l + 1$ bits, so all the status will cost at most
            \[
                l \cdot (\log_2 l + 1)
            \]
            bits.

        \item Input and Output Area.

            There will store the input and output of $M$, and it will cost at most
            \[
                (|x| + |y|) \cdot m
            \]
            bits.
            Where $x$ is the input and $y$ is the output.
    \end{enumerate}

    Now I analysis the comprehensive complexity of UTM. \\

    First, we have to "write" the Machine $M$ and its statuses and inputs, it will cost
    \[
        |P| + l\cdot \log_2 l + l + (|x|+|y|) \cdot m
    \]
    steps.
    And then every time M calculate, the UTM have to access the transition table and modify the
    current status (must go and back);
    so it will cost $2\cdot(|P| + l\cdot \log_2 l+l)$ steps.
    And for every input $x$, $M$ have cost $t_M(x)$ steps until it halts.
    So the UTM have to cost
    \[
        \begin{array}{ll}
            t_U(p\#x) & = t_M(x)\cdot 2\cdot (|P| + l\cdot \log_2 l+l) + |P| + l\cdot \log_2 l + l + (|x|+|y|) \cdot m \\
            &\leqslant t_M(x)\cdot 3|P| + 3|P| \cdot |x| \\
            & = 3|P|(t_M(x)+|x|)
        \end{array}
    \]
    steps.
    So for any specific TM $M$, there is a constant $c=3|P|$, and
    \[
        t_U(x) \leqslant c(t_M(x) + |x|).
    \]

    \\
    \\

    \problem{Prove that there is a function with two values which can be computed by single-tape
    Turing Machine in time $O(n^{100})$, but can not be computed by single-tape Turing Machine in
    time $O(n^{10})$}.\\

    \\
    Solution: \\

    My idea is from the proof of the Time Hierarchy Theorem.
    I will find a function by constructing a special set. \\

    First, assume that there is an appropriate numbering for all Turing Machine,
    for example, according to their binary code and add $1$ to the left side, and let the corresponding
    natural number as the number of Turing Machine.
    Let set A be
    \[
        A = \{1^n | \text{Turing Machine } M_n \text{ halts and return 0 in } n^{10} \text{ steps}.\}
    \]

    According to the time hierarchy theorem, there exists a multi-tapes Turing Machine can accept set
    $A$ in $O(n^{10} \log n)$ time.
    Further, we can build a single-tape Turing Machine which can accept it in $O((n^{10} \log n)^2)$ time.
    Obviously $O(n^{20} \log^2 n) \subset O(n^{100})$, so set $A$ can be identified in time $O(n^{100})$
    but not in time $O(n^{10})$. \\

    Now I build the function by set $A$

    \[
        f(n) =  \begin{cases}
                    1, & \text{if } 1^n \in A. \\
                    0, & \text{otherwise}.
                \end{cases}
    \]

    function $f(n)$ can be compute in time $O(n^{100})$ but not in time $O(n^{10})$ by single-tape
    Turing Machine.



\end{document}