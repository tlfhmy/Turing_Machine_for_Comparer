\documentclass[a4papper]{article}

\usepackage[final]{pdfpages}
\usepackage{graphicx}
\usepackage{amssymb, amsmath, amsthm, mathrsfs}
\usepackage{indentfirst}
\usepackage{amsmath}
\usepackage{listings}
\usepackage{amsfonts}
\usepackage{wasysym}
\usepackage{stmaryrd}
\newtheoremstyle{neosn}{0.5\topsep}{0.5\topsep}{\rm}{}{\sc}{.}{ }{\thmname{#1}\thmnumber{ #2}\thmnote{ {\mdseries#3}}}
\theoremstyle{neosn}
\newtheorem{problem}{Problem}
\newtheorem{example}{Example}

\begin{document}
    \begin{center}
    {\bf Homework 7} \\
        \today \\
        TangLin
    \end{center}

    \problem{Prove that the following problem is \textbf{NP}-complete.\\
    Given: Finite set of natural numbers (in binary). \\
    Problem: Could we separate the set to two parts such that the sums of whose numbers
    are the same?
    } \\

    Solution: \\
    \\
    This problem is very similar with the \textbf{NP}-complete problem SUBSUM SET.
    So I try to reduce this problem from S.S\@. \\
    First, we have to show that this problem is an NP problem.
    We can prove it easily.
    Build a TM $M$ to verify certificate.
    Assume that the original set is $s$, and length its binary representation is $len(s) = n$,
    the certification are two partitions of $s$, denoted by $s_1$ and $s_2$, obviously,
    $len(s_1) + len(s_2) = O(n)$.
    We can easily to verify whether their sums equal.
    Only need to add all the elements and compare two sums.
    These processes obviously can be completed in polynomial time.
    So we know that this problem is in class NP. \\

    Then we show that it is an \textbf{NP}-complete problem by reducing S.S.\ to it.

    Given a set $S$ of integers and a target number $t$, find out a subset $Y \subseteq X$ such
    that the sum of all elements of $Y$ is equal to $t$.
    Let $p$ is the sum of all elements of $X$, then we build a set $X^{\prime} = X \cup \{s - 2t\}$.\\
    Now I prove that if there is a subset $Y \subseteq X$ such that $\sum_{y \in Y}{y} = t $, if and only
    if we can separate $X^{\prime}$ to two parts $X_1$ and $X_2$ such that $\sum_{x \in X_1}{x} = \sum_{x \in X_2}{x}$.\\

    $\Rightarrow$: If there is a subset $Y$ such that $\sum_{y \in Y} y = t$, then we can know that
    $\sum_{x \in X \slash Y}x = s-t$.
    Then we get two sets $Y$ and $X \slash Y$, now we add $s-2t$ to $X$, we get $Y^{\prime}$, and
    $\sum_{y \in Y^{\prime}} y = s-t = \sum_{x \in X \slash Y} x$.
    In addition, $Y^{\prime} \cup X \slash Y = X^{\prime}$, so we can separate $X^{\prime}$ to two
    parts such that their sums are equal. \\

    $\Leftarrow$: If there are a partition of $X^{\prime}$ into two sets $X_1$ and $X_2$ such that the sum over each
    set is the same, i.e. $s-t$, because $\sum_{x \in X^{\prime}}x = \sum_{x \in X} x + s-2t = 2s-2t$.
    And we know that either $s-2t \in X_1$ or $s-2t \in X_2$, with out loss of generality, we can assume
    that $s-2t \in X_1$.
    Now we remove $s-2t$ from $X_1$ and we have $\sum_{x \in X_1 \slash \{s-2t\}}x = s-t - (s-2t) = t$.\\

    Obviously, these operation can execute in polynomial time.
    So we get the result S.S $\leqslant_p$ this problem, and this problem is \textbf{NP}-complete problem. \\

    \\

    \problem{Prove transitivity of Cook's reducibility: Let $f,g,h$ are three partial multivalued
    functions. Let also problem of calculable function $f$ can reduce by Cook to problem of
    calculable function $g$, and $g$ to $h$. Then the problem of calculable function $f$ can be
    reduced by Cook to $h$.} \\

    Solution: \\
    \\
    Assume that $F, G, H$ are the languages respectively generated by function $f, g, h$.
    Because $f$ can be reduced by Cook to $g$, therefore there is TM $M_1$ can ``translate'' every
    element of $F$ to some element of $G$ in polynomial time $p_1(n)$, where $n$ is the length of
    the first element in $F$.
    That is
    \[
        \exists M_1(x) \forall x_f \in F \exists x_g \in G M_1(x_f) = x_g \text{ in }p_1(len(x_f)).
    \]
    Obviously,
    \[
        \exists q_1(n) len(x_g) = q_1(len(x_f)),
    \]
    where $q_1(n)$ is some polynomial function. \\

    Analogously,
    \[
        \exists M_2(x) \forall x_g \in G \exists x_h \in H M_2(x_g) = x_h \text{ in }p_2(len(x_g)).
    \]
    where $p_2(n)$ is some polynomial function.
    And
    \[
        \exists q_2(n) len(x_h) = q_2(len(x_g)),
    \]
    where $q_2(n)$ is some polynomial function. \\

    Now we can construct a TM $M$ by ``connect $M_1$ and $M_2$'', that is, for any input $x$, first
    get $y = M_1(x)$, and then put $y$ as input of $M_2$ and get the final output $M(x) = M_2(y)$,
    \[
        M(x) = M_2(M_1(x)).
    \]
    Obviously, for any string $x_f$ of language $F$, we have
    \[
        M(x_f) = M_2(M_1(x_f)) \in H,
    \]
    because there is some $x_g \in G$ such that $M_1(x_f) = x_g$, and some $x_h \in H$ such that
    $M_2(x_g) = x_h$.
    The first process executed by $M_1$ needs $p_1(len(x_f))$ steps, and the second process
    executed by $M_2$ needs $p_2(len(x_g))$ steps, so
    \[
        p(len(x_f)) = p_2(len(x_g)) + p_1(len(x_f)) = p_2(q_1(len(x_f))) + p_1(len(x_f)).
    \]
    so the function $p$ is also a polynomial function, because all the $p_1, p_2, q_1$ are polynomial.\\

    \\
    So we can say that $f$ can be reduced by Cook to $h$ in polynomial time, i.e.
    \[
        f \leqslant_p g \land g\leqslant_p h \Rightarrow f \leqslant_p h.
    \]

\end{document}